%%%%%%%%%%%%%%%%%%%%%%%%%%%%%%%%%%%%%%%%%%%%%%%%%%%%%%%%%%%%%%%
%
% Welcome to Overleaf --- just edit your LaTeX on the left,
% and we'll compile it for you on the right. If you give
% someone the link to this page, they can edit at the same
% time. See the help menu above for more info. Enjoy!
%
%%%%%%%%%%%%%%%%%%%%%%%%%%%%%%%%%%%%%%%%%%%%%%%%%%%%%%%%%%%%%%%
%\title{Math 453 HW 1}
\documentclass[addpoints]{exam}

\usepackage{amsmath,enumitem,wrapfig}
\usepackage{tikz}
\usepackage{fancyhdr}
\pagestyle{fancy}
\usepackage{hyperref}
\usepackage{amsmath}
\usepackage{color}

\fancyhf{} % clear all fields

\lhead{CPSC 4770/5770 - Large Language Models } % Left header
% \chead{First Line of Center Header \\ Second Line of Center Header} % Center header with two lines
\chead{HW 1}
\rhead{Page \thepage}

\newcommand{\gv}[1]{\ensuremath{\mbox{\boldmath$ #1 $}}} 
\newcommand{\ve}[1]{\mathbf{#1}}

\newcommand{\grad}[1]{\gv{\nabla} #1} % for gradient

\newcommand{\StudentName}{Your name}
\newcommand{\netid}{yyyy8888}
\newcommand{\AssignmentName}{HW 1}

\pagestyle{headandfoot}
\runningheadrule
\firstpageheadrule
\fancyhead[L]{CPSC 488/588 \\ \AssignmentName \\ Name: \StudentName \\ Net id: \netid}

% \firstpageheader{CPSC 488/588}{\StudentName}{\AssignmentName}
% \runningheader{CPSC 488/588}{\StudentName}{\AssignmentName}
\firstpagefooter{}{}{}
\runningfooter{}{}{}

\printanswers

\begin{document}

\begin{center}
\fbox{\fbox{\parbox{5.5in}{
CPSC 4770/5770 - Large Language Models \\
Spring 2026 \\
HW 1 \\

Instructions: \\ Copy this project at \url{https://www.overleaf.com/read/jhtxwqpghsvg#58bc30}. \\
Complete the solutions, and return your solutions in pdf format. Only use latex. \vspace{6pt}\\ 
\textbf{Note:} Please avoid overly verbose answers. \\\vspace{6pt}

Full Name: \StudentName \\
Net ID: \netid}}}
\end{center}
\vspace{3mm}


\begin{questions}

\question[5] \textbf{Q1} \textbf{Warmup.} What is vector semantics and how does it relate to the distributional hypothesis? Name two types of co-occurrence matrices used in vector semantics and briefly describe each.

\begin{solution}[1.3in]

\end{solution}


\question[5]
\textbf{sigmoid} Derive the gradients of the sigmoid function. Then show that it can be rewritten as a function form. That is the gradient of sigmoid should be only represented as a combination of sigmoid functions. As a reminder, the sigmoid function is: 
$\sigma(x) = \frac{1}{1 + e^{-x}}$


\begin{solution}[1in]

\end{solution}


\question[10]
\textbf{word2vec} Recall that the \text{cost loss function } for the word2vec model J: 

$$
J(\theta) = \frac{1}{T} \sum_{t=1}^{T} \sum_{-m \leq j \leq m, j \neq 0} \log p(w_{t+j} | w_t)$$

The output probabilities are also defined as 
$p(o | c) = \frac{\exp(u_o^T v_c)}{\sum_{w=1}^{W} \exp(u_w^T v_c)}$

We derived the gradient for the internal vectors $v_c$. Now derive the gradients for output words $u$.


\begin{solution}[2in]

\end{solution}



\end{questions}
\end{document}